\documentclass[11pt,]{article}
\usepackage[left=1in,top=1in,right=1in,bottom=1in]{geometry}
\newcommand*{\authorfont}{\fontfamily{phv}\selectfont}
\usepackage[]{mathpazo}


  \usepackage[T1]{fontenc}
  \usepackage[utf8]{inputenc}



\usepackage{abstract}
\renewcommand{\abstractname}{}    % clear the title
\renewcommand{\absnamepos}{empty} % originally center

\renewenvironment{abstract}
 {{%
    \setlength{\leftmargin}{0mm}
    \setlength{\rightmargin}{\leftmargin}%
  }%
  \relax}
 {\endlist}

\makeatletter
\def\@maketitle{%
  \newpage
%  \null
%  \vskip 2em%
%  \begin{center}%
  \let \footnote \thanks
    {\fontsize{18}{20}\selectfont\raggedright  \setlength{\parindent}{0pt} \@title \par}%
}
%\fi
\makeatother




\setcounter{secnumdepth}{0}



\title{Buidling Blocks: Understanding Market Functionality  }



\author{\Large \vspace{0.05in} \newline\normalsize\emph{}  }


\date{}

\usepackage{titlesec}

\titleformat*{\section}{\normalsize\bfseries}
\titleformat*{\subsection}{\normalsize\itshape}
\titleformat*{\subsubsection}{\normalsize\itshape}
\titleformat*{\paragraph}{\normalsize\itshape}
\titleformat*{\subparagraph}{\normalsize\itshape}


\usepackage{natbib}
\bibliographystyle{apsr}



\newtheorem{hypothesis}{Hypothesis}
\usepackage{setspace}

\makeatletter
\@ifpackageloaded{hyperref}{}{%
\ifxetex
  \usepackage[setpagesize=false, % page size defined by xetex
              unicode=false, % unicode breaks when used with xetex
              xetex]{hyperref}
\else
  \usepackage[unicode=true]{hyperref}
\fi
}
\@ifpackageloaded{color}{
    \PassOptionsToPackage{usenames,dvipsnames}{color}
}{%
    \usepackage[usenames,dvipsnames]{color}
}
\makeatother
\hypersetup{breaklinks=true,
            bookmarks=true,
            pdfauthor={ ()},
             pdfkeywords = {},  
            pdftitle={Buidling Blocks: Understanding Market Functionality},
            colorlinks=true,
            citecolor=blue,
            urlcolor=blue,
            linkcolor=magenta,
            pdfborder={0 0 0}}
\urlstyle{same}  % don't use monospace font for urls



\begin{document}
	
% \pagenumbering{arabic}% resets `page` counter to 1 
%
% \maketitle

{% \usefont{T1}{pnc}{m}{n}
\setlength{\parindent}{0pt}
\thispagestyle{plain}
{\fontsize{18}{20}\selectfont\raggedright 
\maketitle  % title \par  

}

{
   \vskip 13.5pt\relax \normalsize\fontsize{11}{12} 
\textbf{\authorfont } \hskip 15pt \emph{\small }   

}

}







\begin{abstract}

    \hbox{\vrule height .2pt width 39.14pc}

    \vskip 8.5pt % \small 

\noindent In this article \ldots{}


    \hbox{\vrule height .2pt width 39.14pc}


\end{abstract}


\vskip 6.5pt

\noindent \doublespacing \begin{enumerate}
\def\labelenumi{\arabic{enumi}.}
\tightlist
\item
  Read the paper \emph{What Should Economists Do?} by James M. Buchanan.
\end{enumerate}

\begin{itemize}
\tightlist
\item
  Outline what Buchanan's main point is.
\item
  What does Buchanan mean by the word \textbf{\emph{catallatics}}?
\item
  According to Buchanan what do most economists do?
\item
  What does he think they should do differently?
\end{itemize}

\section{\texorpdfstring{\emph{Answer to
1}}{Answer to 1}}\label{answer-to-1}

Buchanan's main point is that is that economists shouldn't focus on the
simple idea of choice allocation. On page 214, he states ``Only since
\textbf{\emph{The Nature and Significance of Economic Science}}
\citet{buchanan1964} should have economists so exclusively devoted their
energies to the problems raised by scarcity, \ldots{}, and to the
necessity for the making of allocative decisions. He thinks that
economists should focus more of their time on working on catallactics. A
quick Wikipedia search and you can get a rough idea of what this means.
According to the article''Catallactics is a theory of the way the free
market system reaches exchange ratios and prices. It aims to analyze all
actions based on monetary calculation and trace the formation of prices
back to the point where an agent makes his or her choices. It explains
prices as they are, rather than as they ``should'' be." This essentially
means that the economists should not focus on their simplified
equilibrium models, but instead, focus on the broader picture. Buchanan
quotes Adam Smith as the foundation of markets, stating ``is not
originally the effects of any human wisdom, which foresees and intends
that general opulence to which it gives occasion. It is the necessary,
though very slow and gradual, consequence of a certain propensity in
human nature which has in view no such extensive utility; the propensity
to truck, barter, and exchange one thing for another.'' It's this last
phrase that should be focused on. This is what markets are based on, as
alluded by Buchanan. This is what he wants more economists to focus on.
It's this market system that Buchanan refers to as catallactics.

As I have alluded before hand, economists at this point in time focused
way to much on this choice allocation theory. They used perfect
equilibrium models to postulate how a market should be. They replace the
individual with the society and only focus on that. They fail to
understand that it's the individual, exercising their agency to better
themselves, to establish the market under inspection.

Buchanan believes that economists believe that should focus on other
areas, and not hone in on this one specific area of their science. He
insists that more focus be set on how individuals enter into contacts
with each other. In other words, he believes that the idea of how market
structures should be expanded and analyzed. The view of economics should
be broadened to apply to all aspects of life. This is because almost all
interactions can be viewed as economic interactions (with some mixture
of political interaction).

\section{\texorpdfstring{\emph{Abstract}}{Abstract}}\label{abstract}

The purpose of this paper is to aggregate, at least in part, some of the
writings and criticisms of economists. To many economists focus on
simplified, perfect models. They either forget, or ignore what
originally created these simplified markets. All of them, somehow,
stemmed from humanities ``propensity to truck, barter, and exchange one
thing for another'' \citet{buchanan1964should}(quoting Adam Smith). This
paper pulls from several economic disciplines in order nudge economists
to return to their roots of examining markets in order understand, and
postulate what makes them.

\section{\texorpdfstring{\emph{Introduction}}{Introduction}}\label{introduction}

This paper is an attempt, and a rather poor one at that, of aggregating
some of the extensive knowledge of our economic predecessors. Drawing
upon their extensive knowledge, a presentation will be made for why
economists should examine the framework that creates their model. The
idea that Buchanan (buchanan1964should) put forth is that economists
should look at all interactions as a mix of political and economic. They
should not simply be focused on the model, but they should examine what
makes the model: daily interaction among individuals. It was this lack
of examination that Buchanan criticized originally. This paper has the
intent to continue on that noble work and help us all be better to
understand our daily interaction, and how to better model it in our
desire to further understand, and push, the bounds of human ability. As
Buchanan stated ``Learning more about how markets work means learning
more about how markets work.'' \citet{buchanan1964should}

\section{\texorpdfstring{\emph{Outline}}{Outline}}\label{outline}

The gist of the paper will be to examine each reference and build a
framework that will help economists understand that the model is not the
only thing. Especially since most of these models are simplified.
Modeling itself is not the problem, but it's the sole focus on them. The
limits of these models must realized and understood in their
presentations. I don't have the final idea nailed down, but it will
probably go something like

-Intro -discussion of Hayek and Buchanan to set the foundation -Heading
to North to discuss the historical aspects of framework -Discussion on
Ostrom as an example of good practices - One paper from Hedlund for more
examples -Conclusion -Invitation to change and wide spread applicability
(All decisions are economic, unless rent seeking via polilticians
(Buchanan)).

\section{\texorpdfstring{\emph{References}}{References}}\label{references}

\citet{buchanan1987constitution} This Nobel speech will also be used to
help shed light on why it's important to examine and explain the
structure that one is using as an economic model. The model itself is
not enough to explain things. The background and where it comes from is
important too.

\citet{Hayek1989pretence} Will be used to discuss Scientism and expound
upon the original article that started this paper. Will help us
understand that we need to look at the structure of what we are doing,
before we simply dive in with a model.

\citet{ostrom2010beyond} We will use Ostrom to continue to demonstrate
this principle. She advocated for the use of models only after field
work has done. The model should be created after after field work has
been done, and then lab work should be done after the model is created.
This is done to prove that what is done in the field can be replicated
in the lab.

\citet{north1994economic} This paper will be used as another voice in
order to explain the importance of examining the structure of the model
before it is simply used.

\citet{buchanan1964should} This is the foundation of the paper. This is
where the fountain of understanding comes from. It's important to
understand the structure of the model, and see what is actually doing.
See where the outside world creates the model, and don't just assume
it's correct because it is mathematically efficient.

\citet{article}\{Hayek1989pretence, title=\{The pretence of knowledge\},
author=\{Hayek, Friedrich August\}, journal=\{The American Economic
Review\}, volume=\{79\}, number=\{6\}, pages=\{3--7\}, year=\{1989\},
publisher=\{JSTOR\} \}

\citet{article}\{buchanan1987constitution, title=\{The constitution of
economic policy\}, author=\{Buchanan, James M\}, journal=\{The American
economic review\}, volume=\{77\}, number=\{3\}, pages=\{243--250\},
year=\{1987\}, publisher=\{JSTOR\} \}

\citet{article}\{north1994economic, title=\{Economic performance through
time\}, author=\{North, Douglass C\}, journal=\{The American economic
review\}, volume=\{84\}, number=\{3\}, pages=\{359--368\},
year=\{1994\}, publisher=\{JSTOR\} \}

\citet{article}\{ostrom2010beyond, title=\{Beyond markets and states:
polycentric governance of complex economic systems\}, author=\{Ostrom,
Elinor\}, journal=\{American economic review\}, volume=\{100\},
number=\{3\}, pages=\{641--72\}, year=\{2010\} \}

\citet{article}\{buchanan1964should, title=\{What should economists
do?\}, author=\{Buchanan, James M\}, journal=\{Southern Economic
Journal\}, pages=\{213--222\}, year=\{1964\}, publisher=\{JSTOR\} \}

\url{https://en.wikipedia.org/wiki/Catallactics}

\newpage
\singlespacing 
\bibliography{./master}

\end{document}

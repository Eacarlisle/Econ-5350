\documentclass[11pt,]{article}
\usepackage[left=1in,top=1in,right=1in,bottom=1in]{geometry}
\newcommand*{\authorfont}{\fontfamily{phv}\selectfont}
\usepackage[]{mathpazo}


  \usepackage[T1]{fontenc}
  \usepackage[utf8]{inputenc}



\usepackage{abstract}
\renewcommand{\abstractname}{}    % clear the title
\renewcommand{\absnamepos}{empty} % originally center

\renewenvironment{abstract}
 {{%
    \setlength{\leftmargin}{0mm}
    \setlength{\rightmargin}{\leftmargin}%
  }%
  \relax}
 {\endlist}

\makeatletter
\def\@maketitle{%
  \newpage
%  \null
%  \vskip 2em%
%  \begin{center}%
  \let \footnote \thanks
    {\fontsize{18}{20}\selectfont\raggedright  \setlength{\parindent}{0pt} \@title \par}%
}
%\fi
\makeatother




\setcounter{secnumdepth}{0}



\title{Buidling Blocks: Understanding Market Functionality  }



\author{\Large Earl Carlisle\vspace{0.05in} \newline\normalsize\emph{}  }


\date{}

\usepackage{titlesec}

\titleformat*{\section}{\normalsize\bfseries}
\titleformat*{\subsection}{\normalsize\itshape}
\titleformat*{\subsubsection}{\normalsize\itshape}
\titleformat*{\paragraph}{\normalsize\itshape}
\titleformat*{\subparagraph}{\normalsize\itshape}


\usepackage{natbib}
\bibliographystyle{apsr}



\newtheorem{hypothesis}{Hypothesis}
\usepackage{setspace}

\makeatletter
\@ifpackageloaded{hyperref}{}{%
\ifxetex
  \usepackage[setpagesize=false, % page size defined by xetex
              unicode=false, % unicode breaks when used with xetex
              xetex]{hyperref}
\else
  \usepackage[unicode=true]{hyperref}
\fi
}
\@ifpackageloaded{color}{
    \PassOptionsToPackage{usenames,dvipsnames}{color}
}{%
    \usepackage[usenames,dvipsnames]{color}
}
\makeatother
\hypersetup{breaklinks=true,
            bookmarks=true,
            pdfauthor={Earl Carlisle ()},
             pdfkeywords = {},  
            pdftitle={Buidling Blocks: Understanding Market Functionality},
            colorlinks=true,
            citecolor=blue,
            urlcolor=blue,
            linkcolor=magenta,
            pdfborder={0 0 0}}
\urlstyle{same}  % don't use monospace font for urls



\begin{document}
	
% \pagenumbering{arabic}% resets `page` counter to 1 
%
% \maketitle

{% \usefont{T1}{pnc}{m}{n}
\setlength{\parindent}{0pt}
\thispagestyle{plain}
{\fontsize{18}{20}\selectfont\raggedright 
\maketitle  % title \par  

}

{
   \vskip 13.5pt\relax \normalsize\fontsize{11}{12} 
\textbf{\authorfont Earl Carlisle} \hskip 15pt \emph{\small }   

}

}







\begin{abstract}

    \hbox{\vrule height .2pt width 39.14pc}

    \vskip 8.5pt % \small 

\noindent In this article \ldots{}


    \hbox{\vrule height .2pt width 39.14pc}


\end{abstract}


\vskip 6.5pt

\noindent \doublespacing \begin{enumerate}
\def\labelenumi{\arabic{enumi}.}
\tightlist
\item
  Read the paper \emph{What Should Economists Do?} by James M. Buchanan.
\end{enumerate}

\begin{itemize}
\tightlist
\item
  Outline what Buchanan's main point is.
\item
  What does Buchanan mean by the word \textbf{\emph{catallatics}}?
\item
  According to Buchanan what do most economists do?
\item
  What does he think they should do differently?
\end{itemize}

\section{\texorpdfstring{\emph{Answer to
1}}{Answer to 1}}\label{answer-to-1}

Buchanan's main point is that is that economists shouldn't focus on the
simple idea of choice allocation. On page 214, he states ``Only since
\textbf{\emph{The Nature and Significance of Economic Science}} have
economists so exclusively devoted their energies to the problems raised
by scarcity, \ldots{}, and to the necessity for the making of allocative
decisions. He thinks that economists should focus more of their time on
working on catallactics. A quick Wikipedia search and you can get a
rough idea of what this means. According to the article''Catallactics is
a theory of the way the free market system reaches exchange ratios and
prices. It aims to analyze all actions based on monetary calculation and
trace the formation of prices back to the point where an agent makes his
or her choices. It explains prices as they are, rather than as they
``should'' be." This essentially means that the economists should not
focus on their simplified equilibrium models, but instead, focus on the
broader picture. Buchanan quotes Adam Smith as the foundation of
markets, stating ``is not originally the effects of any human wisdom,
which foresees and intends that general opulence to which it gives
occasion. It is the necessary, though very slow and gradual, consequence
of a certain propensity in human nature which has in view no such
extensive utility; the propensity to truck, barter, and exchange one
thing for another.'' It's this last phrase that should be focused on.
This is what markets are based on, as alluded by Buchanan. This is what
he wants more economists to focus on. It's this market system that
Buchanan refers to as catallactics.

As I have alluded before hand, economists at this point in time focused
way to much on this choice allocation theory. They used perfect
equilibrium models to postulate how a market should be. They replace the
individual with the society and only focus on that. They fail to
understand that it's the individual, exercising their agency to better
themselves, to establish the market under inspection.

Buchanan believes that economists believe that should focus on other
areas, and not hone in on this one specific area of their science. He
insists that more focus be set on how individuals enter into contacts
with each other. In other words, he believes that the idea of how market
structures should be expanded and analyzed. The view of economics should
be broadened to apply to all aspects of life. This is because almost all
interactions can be viewed as economic interactions (with some mixture
of political interaction).

\section{Introduction}\label{introduction}

Lorem ipsum. This is a citation of \citet{Buchanan1979}. I will price a
Bermudan swaption with Monte Carlo, because I'm amazing!
\citet{Hayek1945}.

\subsection{Subsection 1}\label{subsection-1}

Blah blah blah

\subsection{Subsection 2}\label{subsection-2}

Yada yada yada. \citet{Figlewski1989} is about simulating option market
maker delta-hedging under conditions of friction, transactions costs,
etc.

\section{Middle Section}\label{middle-section}

More words about computation in finance and economics

\section{Summary and Conclusion}\label{summary-and-conclusion}

Summary of paper. More stuff!

\newpage
\singlespacing 
\bibliography{./master.bib}

\end{document}
